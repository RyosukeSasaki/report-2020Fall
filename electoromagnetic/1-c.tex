\subsubsection*{1-(ウ)}
ラプラス方程式は以下の方程式である.
\begin{align*}
  \laplacian\phi=0
\end{align*}
ただし極座標でのラプラシアンは以下で与えられる.
\begin{align*}
  \laplacian=\frac{1}{r^2}\partial_r(r^2\partial_r)+\frac{1}{r^2\sin\theta}\partial_\theta(\sin\theta\partial_\theta)+\frac{1}{r^2\sin^2\theta}\partial_\varphi^2
\end{align*}
したがって$\phi=R(r)Y(\theta,\varphi)$と変数分離すると
\begin{align*}
  \laplacian\phi=\frac{1}{r^2}\partial_r(r^2\partial_r)RY+\frac{1}{r^2\sin\theta}\partial_\theta(\sin\theta\partial_\theta)RY+\frac{1}{r^2\sin^2\theta}\partial_\varphi^2RY=0
\end{align*}
両辺$r^2$を掛け, $RY$で割る
\begin{align*}
  \frac{r^2}{RY}\cdot\frac{Y}{r^2}\partial_r(r^2\partial_r)R & =-\frac{r^2}{RY}\cdot\frac{R}{r^2\sin\theta}\partial_\theta(\sin\theta\partial_\theta)Y-\frac{r^2}{RY}\cdot\frac{R}{r^2\sin^2\theta}\partial_\varphi^2Y \\
  \frac{1}{R}\partial_r(r^2\partial_r)R                      & =-\frac{1}{Y}\left\{\frac{1}{\sin\theta}\partial_\theta(\sin\theta\partial_\theta)Y+\frac{1}{\sin^2\theta}\partial_\varphi^2Y\right\}
\end{align*}
両辺はそれぞれ$r$と$\theta,\ \varphi$のみに依る関数なので定数$\lambda$とおける.したがって
\begin{empheq}[left=\empheqlbrace]{align}
  \label{equ:1-c-R}
  &\frac{1}{R}\partial_r(r^2\partial_r)R=\lambda \\
  \label{equ:1-c-Y}
  &\frac{1}{Y}\left\{\frac{1}{\sin\theta}\partial_\theta(\sin\theta\partial_\theta)Y+\frac{1}{\sin^2\theta}\partial_\varphi^2Y\right\}=-\lambda
\end{empheq}
さらに$Y(\theta,\varphi)=\Theta(\theta)\varPhi(\varphi)$とすると(\ref{equ:1-c-Y})式は以下のように変数分離できる.
\begin{align*}
  \sin^2\theta\cdot\frac{1}{\Theta\varPhi}\left\{\frac{\varPhi}{\sin\theta}\partial_\theta(\sin\theta\partial_\theta)\Theta+\frac{\Theta}{\sin^2\theta}\partial_\varphi^2\varPhi\right\} & =-\sin^2\theta\lambda                        \\
  \frac{1}{\Theta}\sin\theta\partial_\theta(\sin\theta\partial_\theta\Theta)+\lambda\sin^2\theta                                                                                         & =-\frac{1}{\varPhi}\partial_\varphi^2\varPhi
\end{align*}
これも同様に定数$\kappa$とおけるので
\begin{empheq}[left=\empheqlbrace]{align}
  \label{equ:1-c-Theta}
  &\frac{1}{\Theta}\sin\theta\partial_\theta(\sin\theta\partial_\theta\Theta)+\lambda\sin^2\theta=\kappa\\
  \label{equ:1-c-Phi}
  &\frac{1}{\varPhi}\partial_\varphi^2\varPhi=-\kappa
\end{empheq}
よって(\ref{equ:1-c-R})式, (\ref{equ:1-c-Theta})式, (\ref{equ:1-c-Phi})式がそれぞれの微分方程式となる.