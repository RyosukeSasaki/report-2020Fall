\documentclass[uplatex,a4j,11pt]{jsarticle}
\renewcommand{\thesection}{\arabic{section}.}
\renewcommand{\thesubsubsection}{(\roman{subsubsection})}
\bibliographystyle{jplain}

\usepackage{url}
\usepackage{otf}

\usepackage{amsmath,amsfonts,amssymb}
\usepackage{bm}
\usepackage{siunitx}

\usepackage[dvipdfmx]{graphicx}
\makeatletter
\def\fgcaption{\def\@captype{figure}\caption}
\makeatother
\newcommand{\mfig}[3][width=15cm]{
\begin{center}
\includegraphics[#1]{#2}
\fgcaption{#3 \label{fig:#2}}
\end{center}
}

\begin{document}
第1回 3-(2)
\begin{flushright}
  佐々木良輔
\end{flushright}
関数$f(z)$の特異点を求めその周りでLaurent展開せよ.また,各特異点での留数はいくらか.
\begin{align*}
  f(z)=\cfrac{(z+1)^2}{z(z+3)^2}
\end{align*}
\hrulefill\\
$f(z)$を部分分数分解すると
\begin{align*}
  f(z)=\cfrac{1}{9z}+\cfrac{8}{9(z+3)}-\cfrac{4}{3(z+3)^2}
\end{align*}
特異点は$z=0,-3$である.ここでは$z=-3$周りでLaurent展開する.
\setcounter{subsubsection}{2}
\subsubsection{$0<|z+3|<3$のとき}
$w=z+3$として, $0<|\cfrac{w}{3}|<1$なので
\begin{align*}
  \cfrac{1}{z}&=\cfrac{1}{w-3}\\
  &=-\cfrac{1}{3}\cdot\cfrac{1}{1-\frac{w}{3}}\\
  &=-\cfrac{1}{3}\sum^{\infty}_{n=1}(\cfrac{w}{3})^{n-1}\\
  &=-\cfrac{1}{3}\sum^{\infty}_{n=1}(\cfrac{z+3}{3})^{n-1}
\end{align*}
したがって
\begin{align*}
  f(z)=-\cfrac{4}{3(z+3)^2}+\cfrac{8}{9(z+3)}-\cfrac{1}{27}\sum^{\infty}_{n=1}(\cfrac{z+3}{3})^{n-1}
\end{align*}
となり,これは$z=-3$周りのLaurent級数である.
\subsubsection{$3<|z+3|$のとき}
$w=z+3$として, $0<|\cfrac{3}{w}|<1$なので
\begin{align*}
  \cfrac{1}{z}&=\cfrac{1}{w-3}\\
  &=\cfrac{1}{w}\cdot\cfrac{1}{1-\frac{3}{w}}\\
  &=\cfrac{1}{w}\sum^{\infty}_{n=1}(\cfrac{3}{w})^{n-1}\\
  &=\cfrac{1}{z+3}\sum^{\infty}_{n=1}(\cfrac{3}{z+3})^{n-1}
\end{align*}
したがって
\begin{align*}
  f(z)=-\cfrac{4}{3(z+3)^2}+\cfrac{8}{9(z+3)}+\sum^{\infty}_{n=1}\cfrac{3^{n-3}}{(z+3)^n}
\end{align*}
となり,これは$z=-3$周りのLaurent級数である.\\
また$z=-3$は2位の極なので, $z=-3$での留数は
\begin{align*}
  \underset{z=-3}{\rm{Res}}(f(z))=\cfrac{1}{(2-1)!}\lim_{z\to -3}\left\{\cfrac{d}{dz}(z+3)^2\cfrac{(z+1)^2}{z(z+3)^2}\right\}=\cfrac{8}{9}
\end{align*}
となる.
\end{document}