\documentclass[uplatex,a4j,11pt]{jsarticle}
\bibliographystyle{jplain}

\usepackage{url}

\usepackage{amsmath,amsfonts,amssymb}
\usepackage{bm}
\usepackage{siunitx}

\usepackage[dvipdfmx]{graphicx}
\makeatletter
\def\fgcaption{\def\@captype{figure}\caption}
\makeatother
\newcommand{\setsections}[3]{
\setcounter{section}{#1}
\setcounter{subsection}{#2}
\setcounter{subsubsection}{#3}
}
\newcommand{\mfig}[3][width=15cm]{
\begin{center}
\includegraphics[#1]{#2}
\fgcaption{#3 \label{fig:#2}}
\end{center}
}

\begin{document}
第6回 1-(3)
\begin{flushright}
  佐々木良輔
\end{flushright}
問題:\par
\setcounter{equation}{1}
\begin{align}
  \begin{cases}
    (\Delta\phi-\epsilon\mu\partial^2_t\phi)+\partial_t({\rm div}{\bm A}+\epsilon\mu\partial_t\phi)=-\cfrac{\rho}{\epsilon}\\
    (\Delta{\bm A}-\epsilon\mu\partial^2_t{\bm A})+{\rm grad}({\rm div}{\bm A}+\epsilon\mu\partial_t\phi)=-\mu{\bm i}
  \end{cases}
\end{align}
\begin{align*}
  \phi_L=\phi+\partial_t u_L,\qquad {\bm A}_L={\bm A}-{\rm grad}\ u_L
\end{align*}
式(2)が上のゲージ変換において$u_L$が適当な関係式を満たすとき以下のように書き換わることを示せ.
\begin{align}
  \begin{cases}
    \Delta\phi_L-\epsilon\mu\partial_t^2\phi_L=-\cfrac{\rho}{\epsilon}\\
    \Delta{\bm A}_L-\epsilon\mu\partial_t^2{\bm A}_L=-\mu{\bm i}
  \end{cases}
\end{align}\hrulefill\\
解答:\par
式(2)にゲージ変換を適用すると
\begin{align*}
  \begin{cases}
    (\Delta\phi_L-\epsilon\mu\partial^2_t\phi_L)+\partial_t({\rm div}{\bm A}_L+\epsilon\mu\partial_t\phi_L)=-\cfrac{\rho}{\epsilon}\\
    (\Delta{\bm A}_L-\epsilon\mu\partial^2_t{\bm A}_L)+{\rm grad}({\rm div}{\bm A}_L+\epsilon\mu\partial_t\phi_L)=-\mu{\bm i}
  \end{cases}
\end{align*}
このとき
\begin{align}
  {\rm div}{\bm A}_L+\epsilon\mu\partial_t\phi_L=0\tag{1'}
\end{align}
ならば式(3)となる.式(1')に$\phi_L$, ${\bm A}_L$を代入すると.
\begin{align*}
  {\rm div}{\bm A}-{\rm div\ grad}\ u_L+\epsilon\mu\partial_t(\phi+\partial_t u_L)=0
\end{align*}
${\rm div\ grad}=\Delta$より$u_L$に関する項を左辺に揃えると
\begin{align*}
  (\Delta-\epsilon\mu\partial_t^2)u_L={\rm div}{\bm A}+\epsilon\mu\partial_t\phi\tag{2'}
\end{align*}
よって式(1')を満たす$u_L$が存在し,それが式(2')から求まることがわかる.したがって式(2')のもとで式(2)は式(3)のように書き換わる.式(2')を満たすゲージ変換をLorentz変換と呼ぶ.\par
ただしLorentz変換は以下を満たす$\chi$について自由度が残っている.
\begin{align}
  (\Delta-\epsilon\mu\partial_t^2)\chi=0\tag{3'}
\end{align}
実際に式(2')と辺々足すことで
\begin{align*}
  (\Delta-\epsilon\mu\partial_t^2)(u_L+\chi)={\rm div}{\bm A}+\epsilon\mu\partial_t\phi
\end{align*}
となり$u_L+\chi$という変換のもとで式(3)が得られることがわかる.
\end{document}