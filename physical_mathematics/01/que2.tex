\setcounter{section}{2}
\setcounter{subsection}{0}
\setcounter{equation}{0}
\subsection{}
$n\in\mathbb{N}$において$\rm{Re}(n+\frac{1}{2})>0$なので,漸化式$\bf{\Gamma}(z)=(z-1)\bf{\Gamma}(z-1)$より
\begin{align}
  {\bf\Gamma}(n+\cfrac{1}{2})&=(n-\cfrac{1}{2})(n-\cfrac{3}{2})\cdots\cfrac{3}{2}\cdot\cfrac{1}{2}\cdot{\bf\Gamma}(\cfrac{1}{2})\\
  &=(\cfrac{2n-1}{2})(\cfrac{2n-3}{2})\cdots\cfrac{3}{2}\cdot\cfrac{1}{2}\cdot{\bf\Gamma}(\cfrac{1}{2})\\
  &=\cfrac{(2n-1)!!}{2^n}\cdot{\bf\Gamma}(\cfrac{1}{2})
\end{align}
ここでGamma関数とBeta関数の関係から
\begin{align}
  \begin{split}
    {\bf\Gamma}(\cfrac{1}{2})^2&={\rm{\bf{B}}}(\cfrac{1}{2},\cfrac{1}{2}){\bf\Gamma}(1)\\
    &=\int^1_0t^{-\frac{1}{2}}(1-t)^{-\frac{1}{2}}dt\\
    &=\int^{\frac{\pi}{2}}_0\cfrac{2\sin\theta\cos\theta}{\sin\theta\cos\theta}dt\\
    &=\pi\\
    \therefore\ {\bf\Gamma}(\cfrac{1}{2})&=\sqrt{\pi}
  \end{split}
\end{align}
となる.以上から
\begin{align*}
  {\bf\Gamma}(n+\cfrac{1}{2})=\cfrac{(2n-1)!!}{2^n}\sqrt{\pi}
\end{align*}
\newpage
\subsection{}
\begin{align}
  {\bf\Gamma}(x+1)=\int^{\infty}_0t^x{\rm e}^{-t}dt
\end{align}
において$t=x\tau$, $dt=xd\tau$とすると以下のように書き換えられる.
\begin{align}
  {\bf\Gamma}(x+1)&=\int^{\infty}_0(x\tau)^x{\rm e}^{-x\tau}xd\tau\\
  &=x^{x+1}\int^{\infty}_0\tau^x{\rm e}^{-x\tau}d\tau\\
  &=x^{x+1}\int^{\infty}_0{\rm e}^{-x(\tau-\log\tau)}d\tau\\
  &=x^{x+1}\int^{\infty}_0{\rm e}^{-xh(\tau)}d\tau
\end{align}
ここで$h(\tau)=:\tau-\log\tau$である. $h(\tau)$の増減を考えると,表\ref{tab:zougenhyo}のように$\tau=1$で最小値を取る.
$x\gg1$のとき,図\ref{fig:fig1.eps}のように$e^{-xh(\tau)}$は急激に減衰するため$h(\tau)$の最小値付近が支配的に積分に寄与する.
よって$h(\tau)$を$\tau=1$周りでテイラー展開すると
\begin{align}
  h(\tau)=1+\cfrac{(\tau-1)^2}{2}-\cfrac{(\tau-1)^3}{3}+\cfrac{(\tau-1)^4}{4}-\cdots
\end{align}
したがって
\begin{align}
  {\bf\Gamma}(x+1)&=x^{x+1}\int^{\infty}_0d\tau\ \exp-x\left[1+\cfrac{(\tau-1)^2}{2}-\cfrac{(\tau-1)^3}{3}+\cfrac{(\tau-1)^4}{4}-\cdots\right]\\
  &={\rm e}^{-x}x^{x+1}\int^{\infty}_0d\tau\ \exp-x\left[\cfrac{(\tau-1)^2}{2}-\cfrac{(\tau-1)^3}{3}+\cfrac{(\tau-1)^4}{4}-\cdots\right]
\end{align}
ここで$y=\sqrt{x}(\tau-1)$と変数変換する.このとき$dy=\sqrt{x}dt$,積分区間は$[-\sqrt{x},\infty)$なので
\begin{align}
  {\bf\Gamma}(x+1)&={\rm e}^{-x}x^{x+1}\int^{\infty}_{-\sqrt{x}}\cfrac{dy}{\sqrt{x}}\ \exp-x\left[\cfrac{y^2}{2x}-\cfrac{y^3}{3x\sqrt{x}}+\cfrac{y^4}{4x^2}-\cdots\right]\\
  &={\rm e}^{-x}x^{x+\frac{1}{2}}\int^{\infty}_{-\sqrt{x}}dy\ \exp\left[-\cfrac{y^2}{2}+\cfrac{y^3}{3\sqrt{x}}-\cfrac{y^4}{4x}+\cdots\right]
\end{align}
ただし図\ref{fig:fig2.eps}のように$y\leq-\sqrt{x}$で級数は減衰しているので,積分区間を$(-\infty,\infty)$に拡張し
\begin{align}
  \label{res_2_b}
  {\bf\Gamma}(x+1)&\simeq{\rm e}^{-x}x^{x+\frac{1}{2}}\int^{\infty}_{-\infty}dy\ \exp\left[-\cfrac{y^2}{2}+\cfrac{y^3}{3\sqrt{x}}-\cfrac{y^4}{4x}+\cdots\right]
\end{align}
を得る.
\begin{table}[bhtp]
\caption{増減表}
\label{tab:zougenhyo}
\centering
\begin{tabular}{c|ccc}
$\tau$&$\cdots$&1&$\cdots$\\
\hline
$\frac{dh}{d\tau}$&$-$&0&$+$\\
$h(\tau)$&$\searrow$&1&$\nearrow$
\end{tabular}
\end{table}
\begin{figure}[htbp]
  \begin{minipage}{0.5\hsize}
    \mfig[width=7cm]{fig1.eps}{$e^{-xh(\tau)}$の減衰}
  \end{minipage}
  \begin{minipage}{0.5\hsize}
    \mfig[width=7cm]{fig2.eps}{$x=-\sqrt{x}$と級数との関係}
  \end{minipage}
\end{figure}
\newpage
\subsection{}
(\ref{res_2_b})を再掲する.
\setcounter{equation}{14}
\begin{align}
  {\bf\Gamma}(x+1)&\simeq{\rm e}^{-x}x^{x+\frac{1}{2}}\int^{\infty}_{-\infty}dy\ \exp\left[-\cfrac{y^2}{2}+\cfrac{y^3}{3\sqrt{x}}-\cfrac{y^4}{4x}+\cdots\right]\\
  \label{product}
  &={\rm e}^{-x}x^{x+\frac{1}{2}}\int^{\infty}_{-\infty}dy\ {\rm e}^{-\frac{y^2}{2}}\cdot{\rm e}^{\frac{y^3}{3\sqrt{x}}}\cdot{\rm e}^{-\frac{y^4}{4x}}\cdots
\end{align}
ここで被積分項を$\frac{1}{\sqrt{x}}$について展開し, $\frac{1}{x}$までの項を拾うと
\begin{align}
  {\rm e}^{-\frac{y^2}{2}}\cdot{\rm e}^{\frac{y^3}{3\sqrt{x}}}\cdot{\rm e}^{-\frac{y^4}{4x}}\cdots
  &={\rm e}^{-\frac{y^2}{2}}(1+\frac{y^3}{3\sqrt{x}}+\frac{y^6}{18x}\cdots)(1-\frac{y^4}{4x}+\cdots)\\
  &={\rm e}^{-\frac{y^2}{2}}(1+\frac{y^3}{3\sqrt{x}}-\frac{y^4}{4x}+\frac{y^6}{18x}\cdots)
\end{align}
したがって(\ref{product})は
\begin{align}
  \label{series}
  {\bf\Gamma}(x+1)&\simeq{\rm e}^{-x}x^{x+\frac{1}{2}}\int^{\infty}_{-\infty}dy\ {\rm e}^{-\frac{y^2}{2}}(1+\frac{y^3}{3\sqrt{x}}-\frac{y^4}{4x}+\frac{y^6}{18x}\cdots)
\end{align}
と展開される.ここで以下の積分を考える.
\begin{align}
  {\rm I}_n=\int^{\infty}_{-\infty}x^n{\rm e}^{-\frac{x^2}{2}}dx\qquad(n\in\mathbb{N})
\end{align}
ここで$n:奇数$のとき,被積分関数は奇関数なので
\begin{align}
  {\rm I}_n=0
\end{align}
である.一方で$n:偶数$のとき$m\in\mathbb{N}$を用いて
\begin{align}
  {\rm I}_{2m}&=\int^{\infty}_{-\infty}x^{2m}{\rm e}^{-\frac{x^2}{2}}dx\\
  &=-\left[x^{2m-1}{\rm e}^{-\frac{x^2}{2}}\right]^{\infty}_{-\infty}+(2m-1)\int^{\infty}_{-\infty}x^{2(n-1)}{\rm e}^{-\frac{x^2}{2}}dx\\
  &=(2m-1){\rm I}_{2n-2}\\
  &=(2m-1)!!\int^{\infty}_{-\infty}{\rm e}^{-\frac{x^2}{2}}dx\\
  &=(2m-1)!!\sqrt{2\pi}
\end{align}
したがって(\ref{series})は
\begin{align}
  {\bf\Gamma}(x+1)&\simeq{\rm e}^{-x}x^{x+\frac{1}{2}}\sqrt{2\pi}(1-\frac{1}{4x}\cdot 3+\frac{1}{18x}\cdot 3\cdot 5+\cdots)\\
  &=\sqrt{2\pi}{\rm e}^{-x}x^{x+\frac{1}{2}}(1+\frac{1}{12x}+\cdots)
\end{align}
となる.