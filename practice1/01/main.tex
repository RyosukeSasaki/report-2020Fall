\documentclass[uplatex,a4j,11pt]{jsarticle}
\bibliographystyle{jplain}
\renewcommand{\thesubsubsection}{\arabic{section}-\arabic{subsection}.(\alph{subsubsection})}

\usepackage{url}

\usepackage{amsmath,amsfonts,amssymb}
\usepackage{bm}
\usepackage{siunitx}
\usepackage{titlesec}
\titleformat*{\subsubsection}{\bfseries}

\usepackage[dvipdfmx]{graphicx}
\makeatletter
\def\fgcaption{\def\@captype{figure}\caption}
\makeatother
\newcommand{\setsections}[3]{
\setcounter{section}{#1}
\setcounter{subsection}{#2}
\setcounter{subsubsection}{#3}
}
\newcommand{\mfig}[3][width=15cm]{
\begin{center}
\includegraphics[#1]{#2}
\fgcaption{#3 \label{fig:#2}}
\end{center}
}
\def\product<#1>{\langle #1 \rangle}

\begin{document}
\title{物理学演習第1 No.1, 2レポート}
\author{佐々木良輔}
\date{}
\maketitle
\setsections{1}{1}{3}
\subsubsection{}
関数$f(x)=x^2$をFourier級数で表わせ\\
$f(x)$は偶関数なので
\begin{align*}
  B_n&=0\\
  A_0&=\frac{1}{\pi}\int^{\pi}_{-\pi}x^2{\rm d}x=\frac{2}{3}\pi^2\\
  A_n&=\frac{1}{\pi}\int_{-\pi}^{\pi}x^2\cos nx{\rm d}x\\
  &=\frac{1}{\pi}\biggl\{\biggl[x^2\frac{\sin nx}{n}\biggr]^{\pi}_{-\pi}-\frac{2}{n}\int^{\pi}_{-\pi}x\sin nx{\rm d}x\biggr\}\\
  &=-\frac{2}{n\pi}\biggl\{\biggl[-x\frac{\cos nx}{n}\biggr]^{\pi}_{-\pi}+\frac{1}{n}\int^{\pi}_{-\pi}\cos nx{\rm d}x\biggr\}\\
  &=\frac{2}{n^2\pi}(\pi\cos n\pi+\pi\cos n\pi)\\
  &=\frac{4}{n^2}\cos n\pi\\
  &=\frac{4}{n^2}(-1)^n
\end{align*}
以上から
\begin{align}
  \label{1-1-d}
  f(x)=\frac{\pi^2}{3}+4\sum^{\infty}_{n=1}\frac{(-1)^n}{n^2}\cos nx
\end{align}
\subsubsection{}
\noindent(i)\\
(\ref{1-1-d})式において$x=\pi$とすると
\begin{align*}
  f(\pi)&=\frac{\pi^2}{3}+4\sum^{\infty}_{n=1}\frac{(-1)^n}{n^2}\cos n\pi\\
  &=\frac{\pi^2}{3}+4\sum^{\infty}_{n=1}\frac{(-1)^{2n}}{n^2}=\pi^2\\
  &\therefore\zeta(2)=\sum^{\infty}_{n=1}\frac{1}{n^2}=\frac{\pi^2}{6}
\end{align*}
(ii)\\
(\ref{1-1-d})式において$x=0$とすると
\begin{align*}
  f(0)=\frac{\pi^2}{3}+4\sum^{\infty}_{n=1}\frac{(-1)^n}{n^2}=0
\end{align*}
両辺に$-1$を掛けると
\begin{align*}
  -\frac{\pi^2}{3}+4\sum^{\infty}_{n=1}\frac{(-1)^{n+1}}{n^2}&=0\\
  -{\rm Li}_2(-1)=\sum^{\infty}_{n=1}\frac{(-1)^{n+1}}{n^2}&=\frac{\pi^2}{12}
\end{align*}
\setsections{2}{3}{3}
\subsubsection{}
$A=x$, $B=p$としたとき交換関係を考える.
\begin{align*}
  [x,p]\psi&=x\frac{\hbar}{i}\frac{\rm d}{{\rm d}x}\psi-\frac{\hbar}{i}\frac{\rm d}{{\rm d}x}x\psi\\
  &=x\frac{\hbar}{i}\frac{\rm d}{{\rm d}x}\psi-\left(\frac{\hbar}{i}\frac{\rm d}{{\rm d}x}x\cdot\psi+\frac{\hbar}{i}\frac{\rm d}{{\rm d}x}\psi\cdot x\right)\\
  &=-\frac{\hbar}{i}\psi=i\hbar\psi
\end{align*}
したがって$[x,p]=i\hbar$であり$C=\hbar$となる.
以上とKennard-Robertsonの不等式から
\begin{align*}
  \Delta x\Delta p\geq\frac{\hbar}{2}
\end{align*}
が示された.

また等号成立時は$I(\lambda)=\product<(\lambda\tilde{A}+i\tilde{B})\psi,(\lambda\tilde{A}+i\tilde{B})\psi>=0$より$(\lambda\tilde{A}+i\tilde{B})\psi=0$である.
ここで
\begin{align*}
  \tilde{x}&=x-\product<x>\\
  \tilde{p}&=\frac{\hbar}{i}\frac{\rm d}{{\rm d}x}-\product<p>\\
\end{align*}
なので$(\lambda\tilde{A}+i\tilde{B})\psi=0$は以下のように書き換わる.
\begin{align*}
  \left[\lambda(x-\product<x>)+\hbar\frac{\rm d}{{\rm d}x}-i\product<p>\right]\psi=0
\end{align*}
この微分方程式を解く.
\begin{align*}
  \frac{{\rm d}\psi}{{\rm d}x}&=\left[-\frac{\lambda}{\hbar}(x-\product<x>)+\frac{i}{\hbar}\product<p>\right]\psi
\end{align*}
変数分離法を用いると
\begin{align*}
  \int\frac{{\rm d}\psi}{\psi}&=\int{\rm d}x\left(-\frac{\lambda}{\hbar}(x-\product<x>)+\frac{i}{\hbar}\product<p>\right)\\
  \log\psi&=-\frac{\lambda}{2\hbar}(x-\product<x>)^2+\frac{i}{\hbar}\product<p>x+C\\
  \psi&=\exp\left((x-\product<x>)^2+\frac{i}{\hbar}\product<p>x+C\right)\\
  &=\exp\left((x-\product<x>)^2+\frac{i}{\hbar}\product<p>x+C\right)\\
\end{align*}
\begin{align*}
  \therefore\psi(x)&=A\exp\left(\left(x-\frac{2\product<x>+\frac{i}{\hbar}\product<p>}{2}\right)^2\right)\qquad
  (A:={\rm e}^{\product<x>^2+C-\left(\frac{2\product<x>+\frac{i}{\hbar}\product<p>}{2}\right)^2})
\end{align*}
となり,これは確かにガウス関数である.
\end{document}