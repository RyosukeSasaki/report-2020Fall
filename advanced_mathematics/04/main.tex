\documentclass[uplatex,a4j,11pt]{jsarticle}
\bibliographystyle{jplain}

\usepackage{url}

\usepackage{amsmath,amsfonts,amssymb}
\usepackage{bm}
\usepackage{siunitx}

\usepackage[dvipdfmx]{graphicx}
\makeatletter
\def\fgcaption{\def\@captype{figure}\caption}
\makeatother
\newcommand{\setsections}[3]{
\setcounter{section}{#1}
\setcounter{subsection}{#2}
\setcounter{subsubsection}{#3}
}
\newcommand{\mfig}[3][width=15cm]{
\begin{center}
\includegraphics[#1]{#2}
\fgcaption{#3 \label{fig:#2}}
\end{center}
}

\begin{document}
学籍番号 61908697\qquad 物理学科\qquad 佐々木良輔\\
(1)\\
周期を$[-\frac{T_0}{2},\frac{T_0}{2}]$とする.パルス幅が$d$のとき
\begin{align}
  C_0&=\frac{1}{T_0}\int^{\frac{T_0}{2}}_{-\frac{T_0}{2}}f(t){\rm d}t\nonumber\\
  &=\frac{1}{T_0}\int^{\frac{d}{2}}_{-\frac{d}{2}}A{\rm d}t\nonumber\\
  &=\frac{Ad}{T_0}\\
  C_n&=\frac{1}{T_0}\int^{\frac{T_0}{2}}_{-\frac{T_0}{2}}f(t){\rm e}^{-in\omega_0t}{\rm d}t\nonumber\\
  &=\frac{A}{T_0}\int^{\frac{d}{2}}_{-\frac{d}{2}}{\rm e}^{-in\omega_0t}dt\nonumber\\
  &=\frac{2A}{T_0n\omega_0}\frac{1}{2i}({\rm e}^{in\omega_0t}-{\rm e}^{-in\omega_0t})\nonumber\\
  &=\frac{2A}{T_0n\omega_0}\sin\Bigl(\frac{n\omega_0d}{2}\Bigr)\nonumber\\
  &=\frac{Ad}{T_0}\frac{\sin\Bigl(\frac{n\omega_0d}{2}\Bigr)}{\frac{n\omega_0d}{2}}
\end{align}
したがって$T_0=\frac{1}{4}$, $d=\frac{1}{20}$のとき振幅スペクトル$|C_n|$は下図のようになる.
\mfig[width=10cm]{fig1.eps}{振幅スペクトル}
\newpage
(2)\\
(1)の結果を用いる.$T_0=\frac{1}{2}$, $d=\frac{1}{20}$のとき振幅スペクトル$|C_n|$は下図のようになる.
\mfig[width=10cm]{fig2.eps}{振幅スペクトル}
\end{document}