\subsection*{1-4}\noindent
(1)図\ref{fig:fx_ax.png}や図\ref{fig:ax.png}から級数の次数が高いほど元の波形に近づいていることがわかる.そして無限級数となったとき,級数は元の波形と一致する.(完全性)\\
(2)図\ref{fig:cx.png}や図\ref{fig:fx_ax.png}の不連続点をみると関数の不連続点$x=x_0$では級数の値は
\begin{align*}
  \frac{\displaystyle\lim_{x\rightarrow-x_0}f(x)+\displaystyle\lim_{x\rightarrow+x_0}f(x)}{2}
\end{align*}
に収束している.(Dirichletの定理)\\
(3)図\ref{fig:fx_ax.png},図\ref{fig:delta_ax.png}を見ると,不連続点では誤差が大きくなっていることがわかる.不連続点では項数が有限である限り級数は収束しない.(Gibbsの現象)\\
(4)表\ref{tbl:b},表\ref{tbl:c}を見ると$b(x)$の方が$c(x)$に比べて係数がすぐに小さくなっていることがわかる.
そのため図\ref{fig:bx.png},図\ref{fig:cx.png}のように, $b(x)$の方が$c(x)$より早く元の関数に近づいていることがわかる.
\mfig[width=10cm]{coeff.png}{$b(x)$と$c(x)$の係数}
(5)図\ref{fig:bx.png},図\ref{fig:cx.png}からFourier余弦級数は偶関数, Fourier正弦級数は奇関数となっていることがわかる.