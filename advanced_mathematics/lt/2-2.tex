\subsection*{2-2}
元データは幾つかの余弦波の合成であるので,理想的には周波数スペクトルは図\ref{fig:spec_example.png}のように$\delta$関数からなるはずである.
しかし実際には,図\ref{fig:data/kadai2/Spectrum-kadai2.png}のように量子化誤差や離散化によりノイズやサイドローブが乗っていると考えられる.
したがって振幅スペクトルのピークを拾うことでノイズを除去し,元の余弦波を得る.
ピークはScipyライブラリのsignal.find\_peaks関数による極大値,ならびに閾値により検出している.
\mfig[width=6cm]{spec_example.png}{理想的なスペクトルの例}
図\ref{fig:data/kadai2/Amplitude-kadai2.png}に振幅スペクトル,図\ref{fig:data/kadai2/Phase-kadai2.png}に位相スペクトルを示す.
それぞれのスペクトルには振幅スペクトルのピークを書き込んでいる.
なお振幅は以下のように計算される.
\begin{align*}
  A_k=\frac{|F_k|\times 2}{N}
\end{align*}
また表\ref{tab:2-2}にはノイズ除去により得られたピーク周波数と,そこでの振幅,位相を示す.
これを元に再合成した波形を図\ref{fig:data/kadai2/Wave-kadai2.png}に示す.図\ref{fig:data/kadai2/Wave-kadai2.png}のように
再合成した波形は元データをある程度再現している.したがって元データは表\ref{tab:2-2}に示した周波数,振幅,位相の余弦波の重ね合わせであると考えられる.
\mfig[width=10cm]{data/kadai2/Amplitude-kadai2.png}{振幅スペクトルとそのピーク}
\mfig[width=10cm]{data/kadai2/Phase-kadai2.png}{位相スペクトル}
\mfig[width=10cm]{data/kadai2/Wave-kadai2.png}{再合成した波形}
\begin{table}[h]
  \caption{ピーク周波数,振幅,位相}
  \label{tab:2-2}
  \centering
  \begin{tabular}{ccc}
    \hline
    周波数 / $\si{\hertz}$ & 振幅  & 位相 / $\si{\degree}$ \\
    \hline \hline
    40.0                   & 1.20  & 88.2                  \\
    50.0                   & 1.33  & -88.4                 \\
    120                    & 1.04  & -5.86                 \\
    130                    & 1.09  & -88.2                 \\
    220                    & 0.197 & -38.9                 \\
    260                    & 0.843 & -65.9                 \\
    380                    & 0.548 & -84.4                 \\
    \hline
  \end{tabular}
\end{table}
