\subsection*{2-1}
図\ref{fig:data/kadai2/WaveOrigin-kadai2.png},図\ref{fig:data/kadai2/Spectrum-kadai2.png}に元波形と周波数スペクトルを示す.これらは自作のpythonスクリプト(\ref{script})で作成した.
pythonスクリプトではcsvデータを配列に格納し,これにnumpyライブラリのfft.fft関数を掛けている.
この関数は以下の定義式に基づき離散Fourier変換を行う.\cite{Discrete33:online}
\begin{align*}
  F_k=\sum_{m=0}^{n-1}f_k\exp\left(-2\pi i\frac{mk}{n}\right)
\end{align*}
この値は複素数であるので,スペクトルにおいては絶対値を取った.結果はmatplotlibにて出力した.
\mfig[width=10cm]{data/kadai2/WaveOrigin-kadai2.png}{元波形}
\mfig[width=10cm]{data/kadai2/Spectrum-kadai2.png}{周波数スペクトル}