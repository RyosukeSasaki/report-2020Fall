\subsection*{3-1}
図\ref{fig:data/kadai3-1/Wave-kadai3.png}に元データと再合成した波形,
図\ref{fig:data/kadai3-1/Spectrums-kadai3.png}に周波数スペクトル,
図\ref{fig:data/kadai3-1/Amplitudes-kadai3.png}に振幅スペクトルを示す.
表\ref{tab:3-1}に各ピークの周波数,振幅,位相を示す.
Nyquist周波数$f_{N}=64.0\ \si{\hertz}$である.
\mfig[width=10cm]{data/kadai3-1/Wave-kadai3.png}{元データと再合成したデータ}
\mfig[width=10cm]{data/kadai3-1/Spectrums-kadai3.png}{周波数スペクトル}
\mfig[width=10cm]{data/kadai3-1/Amplitudes-kadai3.png}{振幅スペクトル}
\begin{table}[h]
\caption{各ピークの振幅}
\label{tab:3-1}
\centering
\begin{tabular}{ccc}
\hline
周波数 / $\si{\hertz}$ & 振幅  & 位相 / $\si{\degree}$ \\
\hline \hline
4.00&1.00&$2.03\times10^{-14}$\\
28.0&1.00&$1.80\times10^2$\\
40.0&0.500&$-9.00\times10^1$\\
\hline
\end{tabular}
\end{table}