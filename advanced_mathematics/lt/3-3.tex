\subsection*{3-3}
図\ref{fig:data/kadai3-3/Wave-kadai3.png}に元データと再合成した波形,
図\ref{fig:data/kadai3-3/Spectrums-kadai3.png}に周波数スペクトル,
図\ref{fig:data/kadai3-3/Amplitudes-kadai3.png}に振幅スペクトルを示す.
表\ref{tab:3-3}に各ピークの周波数,振幅,位相を示す.
Nyquist周波数$f_{N}=16.0\ \si{\hertz}$である.
また表\ref{tab:zj}にはダウンサンプリング後のデータを示した.

元データでの$4.00\ \si{\hertz}$と$28.0\ \si{\hertz}$の余弦波は$f_N$で折り返したときにちょうど重なっている.
またそれぞれの余弦波は振幅が等しく逆位相であったので,エイリアシングノイズと$4.00\ \si{\hertz}$の余弦波はちょうど打ち消し合うと考えられる.
また$40.0\ \si{\hertz}$の余弦波が折り返すことで$\pm F_N-(\pm40\mp F_N)=\mp8\ \si{\hertz}$の余弦波が生じると考えられる.
実際にダウンサンプリング後の振幅スペクトルには$8.00\ \si{\hertz}$のピークのみが現れ,他の成分は消えている.
さらにダウンサンプリングによりデータ数が$1/4$になったため,周波数スペクトル$|Z_k|$の値も$1/4$になっている.
\mfig[width=10cm]{data/kadai3-3/Wave-kadai3.png}{元データと再合成したデータ}
\mfig[width=10cm]{data/kadai3-3/Spectrums-kadai3.png}{周波数スペクトル}
\mfig[width=10cm]{data/kadai3-3/Amplitudes-kadai3.png}{振幅スペクトル}
\begin{table}[h]
\caption{各ピークの振幅}
\label{tab:3-3}
\centering
\begin{tabular}{ccc}
\hline
周波数 / $\si{\hertz}$ & 振幅  & 位相 / $\si{\degree}$ \\
\hline \hline
$8.00$&0.500&$-9.00\times10^{2}$\\
\hline
\end{tabular}
\end{table}